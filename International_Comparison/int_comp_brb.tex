\documentclass[ngerman,a4paper]{article}
\usepackage{lmodern}
\usepackage{amssymb,amsmath}
\usepackage{ifxetex,ifluatex}
\usepackage{fixltx2e} % provides \textsubscript
\ifnum 0\ifxetex 1\fi\ifluatex 1\fi=0 % if pdftex
  \usepackage[T1]{fontenc}
  \usepackage[utf8]{inputenc}
\else % if luatex or xelatex
  \ifxetex
    \usepackage{mathspec}
  \else
    \usepackage{fontspec}
  \fi
  \defaultfontfeatures{Ligatures=TeX,Scale=MatchLowercase}
  \newcommand{\euro}{€}
\fi
% use upquote if available, for straight quotes in verbatim environments
\IfFileExists{upquote.sty}{\usepackage{upquote}}{}
% use microtype if available
\IfFileExists{microtype.sty}{%
\usepackage{microtype}
\UseMicrotypeSet[protrusion]{basicmath} % disable protrusion for tt fonts
}{}
\usepackage[margin=1in]{geometry}
\usepackage{hyperref}
\PassOptionsToPackage{usenames,dvipsnames}{color} % color is loaded by hyperref
\hypersetup{unicode=true,
            pdftitle={Vergütung von hausräztlichen Leistungen},
            pdfauthor={Beat Brüngger},
            pdfsubject={Die Situation in der Schweiz im Vergleich zu Ansätzen im benachbarten
Ausland},
            pdfborder={0 0 0},
            breaklinks=true}
\urlstyle{same}  % don't use monospace font for urls
\ifnum 0\ifxetex 1\fi\ifluatex 1\fi=0 % if pdftex
  \usepackage[shorthands=off,main=ngerman]{babel}
\else
  \usepackage{polyglossia}
  \setmainlanguage[variant=swiss]{german}
\fi
\usepackage{longtable,booktabs}
\usepackage{graphicx,grffile}
\makeatletter
\def\maxwidth{\ifdim\Gin@nat@width>\linewidth\linewidth\else\Gin@nat@width\fi}
\def\maxheight{\ifdim\Gin@nat@height>\textheight\textheight\else\Gin@nat@height\fi}
\makeatother
% Scale images if necessary, so that they will not overflow the page
% margins by default, and it is still possible to overwrite the defaults
% using explicit options in \includegraphics[width, height, ...]{}
\setkeys{Gin}{width=\maxwidth,height=\maxheight,keepaspectratio}
\setlength{\parindent}{0pt}
\setlength{\parskip}{6pt plus 2pt minus 1pt}
\setlength{\emergencystretch}{3em}  % prevent overfull lines
\providecommand{\tightlist}{%
  \setlength{\itemsep}{0pt}\setlength{\parskip}{0pt}}
\setcounter{secnumdepth}{5}

%%% Use protect on footnotes to avoid problems with footnotes in titles
\let\rmarkdownfootnote\footnote%
\def\footnote{\protect\rmarkdownfootnote}

%%% Change title format to be more compact
\usepackage{titling}

% Create subtitle command for use in maketitle
\newcommand{\subtitle}[1]{
  \posttitle{
    \begin{center}\large#1\end{center}
    }
}

\setlength{\droptitle}{-2em}
  \title{Vergütung von hausräztlichen Leistungen}
  \pretitle{\vspace{\droptitle}\centering\huge}
  \posttitle{\par}
\subtitle{Die Situation in der Schweiz im Vergleich zu Ansätzen im benachbarten
Ausland}
  \author{Beat Brüngger}
  \preauthor{\centering\large\emph}
  \postauthor{\par}
  \predate{\centering\large\emph}
  \postdate{\par}
  \date{Sept. 8, 2016}


\usepackage{fancyhdr}
\pagestyle{fancy}
\fancyhead{}
\renewcommand{\headrulewidth}{0pt}
\fancyfoot[C]{Vergütung von hausräztlichen Leistungen}
\fancyfoot[LE,RO]{\thepage}

% Redefines (sub)paragraphs to behave more like sections
\ifx\paragraph\undefined\else
\let\oldparagraph\paragraph
\renewcommand{\paragraph}[1]{\oldparagraph{#1}\mbox{}}
\fi
\ifx\subparagraph\undefined\else
\let\oldsubparagraph\subparagraph
\renewcommand{\subparagraph}[1]{\oldsubparagraph{#1}\mbox{}}
\fi

\begin{document}
\maketitle
\begin{abstract}
Das in der Schweiz bei der Vergütung von ambulanten
Gesundheitsleistungen vewendete Tarifsystem TARMED wurde 2004
eingeführt, und bedarf anerkannterweise einer Reform. Die darin
aufgeführten Einzelleistungen haben aus heutiger Sicht vielfach keinen
sachgerechten Preis mehr, was zu Fehlanreizen führt. Aufgrund der
diametral entgegengesetzten Interessen der beiden Tarifpartner
Ärzteschaft und Krankenversicherer sind die Verhandlungen rund um die
Tarifrefom seit Jahren blockiert. Ausserdem steigen die
Gesundheitsausgaben in einem so starken Ausmass, dass auf politischer
Ebene Lösungen gefordert werden, welche das Gesundheitssystem
langfristig finanzierbar halten. Steigende Gesundheitskosten sind auch
in den benachbarten Ländern ein vieldiskutiertes Thema. Aus diesem Grund
sollen die in anderern Ländern vorhandenen Lösungsansätze für dieses
Problem auf deren Tauglichkeit für die Anwendung in der Schweiz
überprüft werden.
\end{abstract}


\newpage

{
\setcounter{tocdepth}{2}
\tableofcontents
}
\pagebreak

\section{Einführung}\label{einfuhrung}

Die Vergütung von ambulanten ärztlichen Leistungen wird in der Schweiz
über das Tarifwerk TARMED (\textbf{\emph{tar}}if
\textbf{\emph{méd}}ical) geregelt, sowohl bei Grundversorgern, als auch
bei Spezialärzten. Der TARMED wurde per 1.1.2004 eingeführt, und ist ein
Einzelleistungstarif (engl.: fee-for-service, FFS). Das bedeutet, dass
anerkannte Leistungserbringer für das Erbringen jeder im TARMED
aufgeführten Leistung einen im Voraus bekannten Preis verrechnen
können.\\
Wie bereits in der von der Eidgenössischen Finanzkontrolle EFK 2010
erstellten Evaluation der Zielerreichung des TARMED (Sangra und Courbat
2010) beschrieben, bietet die aktuelle Implementation dieses Systems
verschiedene Schwachstellen. Insbesondere die ebenda auf Seite 92
beschriebenen Fehlanreize durch zu tief resp. zu hoch bewerteten
Leistungen führen in der Schweiz aktuell zu Problemen in der Vergütung
von ambulanten Gesundheitsleistungen. Der Bericht führt unter anderen
folgende beiden Punkte auf:

\begin{itemize}
\tightlist
\item
  Steigende Kosten in den Sozialversicherungen durch vermehrte
  Abrechnung von zu hoch bewerteten Leistungen, ``obwohl sie evtl.
  medizinisch nicht unbedingt notwendig wären.''
\item
  Ungleiche Einkommensentwicklung, je nach Spezialität der Ärzte, da
  operativ tätige Ärzte mittels Verbesserung der Effizienz einen
  grösseren Handlungsspielraum haben ihr Einkommen zu steigern, als dies
  Grundversorger haben.
\end{itemize}

Diese beiden Punkte bedeuten, dass die vorhandenen Fehlanreize im
Einzelleistungstarif TARMED potentiell dazu führen, dass eine
Mengenausweitung an medizinischen Leistungen einen stärker als nötigen
Anstieg der Gesundheitskosten zur Folge hat, teilweise medizinisch nicht
sinnvolle Behandlungen durchgeführt werden, und dass weniger Ärzte in
Ausbildung sich für eine Spezialisierung in Hausarztmedizin
entschliessen, als dies für eine flächendeckende Versorgung nötig wäre.
Indbesonderer in ländlichen Regionen der Schweiz ist die Dichte an
Grundversorgern tiefer, als erwünscht ist.\\
Aktuell werden steigenden Gesundheitskosten und der Mangel an Hausärzten
im Zuge der zum wiederholten Male gescheiterten Revision des TARMED, und
der bevorstehenden Bekanngabe der Prämien der Grundversicherung OKP für
das kommende Jahr, auch in den Medien diskutiert (z.B. in Forster (2016)
und Nordwestschweiz (2016)).\\
Im Masterplan ``Gesundheit 2020'' (EDI 2013) adressiert das
Eidgenössische Departement des Inneren (EDI) zudem die obigen Punkte im
``Ziel 2.2: Gesundheit durch Effizienzsteigerungen bezahlbar halten''.
Gemäss dem EDI müssen hierzu Fehlanreize im Vergütungssystem der
ambulanten Versorgung (unter anderen: TARMED) beseitigt werden, und
Massnahmen zur Effizienzsteigerung und Massnahmen zur
Qualitätssteigerung sich gegenseitig unterstützen und gemeinsam geplant
werden. Konkretisiert wird dieses Ziel mit einer dazu zugeordneten
Massnahme: ``Stärkung der Pauschalabgeltungen gegenüber den
Einzelleistungstarifen sowie Revision bestehender Tarife (bspw. TARMED),
um die Anreize zur Ausdehnung des Angebotes bei den
Leistungserbringenden einzuschränken.''

\section{Vorgehensweise}\label{vorgehensweise}

Die Literaturrechereche wurde nicht systematisch durchgeführt. Neben
Dokumenten der eidg. Verwaltung und des European Observatory on Health
Systems and Policies sind über \href{scholar.google.ch}{Google Scholar}
gefundene Dokumente in diese Arbeit eingeflossen.

\begin{itemize}
\tightlist
\item
  Für die Beurteilung der aktuellen Situation in der Schweiz wird auf
  Dokumente der eidg. Verwaltung (Sangra und Courbat (2010) und EDI
  (2013)) zurückgegriffen.\\
\item
  Das European Observatory on Health Systems and Policies zeigt in
  Thomson, Foubister, und Mossialos (2009) die Palette der in der
  Europäischen Union in Gebrauch befindlichen Vergütungssysteme auf.\\
\item
  Anhand des Fallbeispiels Niederlande, beschrieben in Smolders und
  Seinen (2007), werden die Vor- und Nachteile unterschiedlicher
  Vergütungssysteme angeschaut, und das System in den Niederlanden
  evaluiert.
\end{itemize}

Aufgrund dieser Auslegeordnung werden schlussendlich mögliche
Anpassungen im in der Schweiz implementierten System diskutiert.

\section{Resultate und Diskussion}\label{resultate-und-diskussion}

\subsection{Vergütungssystem in der
Schweiz}\label{vergutungssystem-in-der-schweiz}

In der Schweiz kommt in der Vergütung hausärztlicher Leistungen mit dem
Tarifwerk TARMED ein Einzelleistungstarif (FFS) zur Anwendung. Jede von
einem anerkannten Leistungserbringer erbrachten Leistung hat einen je
nach Kanton unterschiedlichen Preis. Dieser kann dem Patienten resp.
dessen Krankenversicherung in Rechnung gestellt werden.\\
Daneben kommt in einigen Versicherungsmodellen (üblicherweise unter der
Bezeichnung Managed Care bekannt), welche Krankenversicherungen
freiwillig anbieten können, ein Capitationsystem zur Anwendung. Hierbei
wird mit einem Netzwerk von Ärzten aufgrund des ihm zugeordneten
Versichertenkollektivs ein (teilleiweise risikoadjustiertes) Budget
vereinbart. Benötigt das Ärztenetzwerk für die Behandlung des
Versichertenkollektivs in einem Jahr weniger als das vereinbarte Budget,
erhält es von der Versicherung einen Teil des gegenüber dem Budget
eingesparten Betrags als Bonus ausbezahlt. Bei Überschreitung des
Budgets muss das Netzwerk der Versicherung möglicherweise einen Malus
zahlen. Die genauen Modalitäten dieser Vereinbarungen werden zwischen
den Versicherungen und den Ärztenetzwerken verhandelt, und variieren
deshalb. Die Versicherten erhalten für die Teilnahme an einem solchen
Versicherungsmodell einen Rabatt auf ihre Versicherungsprämie.\\
Somit kommt in der Schweiz ein gemischtes System, bestehend aus FFS und
Capitation mit Budgetverantwortung, zur Anwendung. Gundlange für das
Einkommen der Grundversoger bildet dabei immer der Einzelleistungstarif.
Das Capitationsystem modifiziert über die Zahlungen des Ärztenetzwerks
allenfalls die Einnahmen der einzelnen Ärzte.

\subsection{Überblick über verschiedene Vergütungssysteme in der
EU}\label{uberblick-uber-verschiedene-vergutungssysteme-in-der-eu}

Wie die folgende Tabelle aus Thomson, Foubister, und Mossialos (2009),
Seiten 42-43, zeigt, kommt zum Zeitpunkt der Studie bei der Vergütung
von Grundversorgern in den meisten Ländern der EU, wie auch in der
Schweiz, ein System mit einer Mischung aus FFS und Capitation zur
Anwendung. Insbesondere einige der neueren Länder der EU haben ihr
Gesundheitssystem weg von fixen Salären (welche ebenfalls weiter
vorkommen) und hin zu FFS resp. Capitation reformiert, seltener mit
zusätzlichen performance-abhängigen Zahlungen (Thomson, Foubister, und
Mossialos (2009), Seiten 85-86).\\
Der Reformprozess ist auch in vielen EU-Mitgliedsstaaten noch nicht
abgeschlossen, wobei hauptsächlich weiter ausdifferenzierte
Vergütungssysteme, welche die erbrachte Menge an Leistungen und deren
Qualität miteinbezieht, folgen sollen. Eine Verlinkung von
performance-abhängigen Zahlungen mit dem gesundheitsrelevanten Ergebnis
der Behandlung soll gemäss Thomson, Foubister, und Mossialos (2009),
Seite 92, weiter vorangetrieben werden.

\begin{longtable}[c]{@{}ll@{}}
\toprule
\begin{minipage}[b]{0.08\columnwidth}\raggedright\strut
Land
\strut\end{minipage} &
\begin{minipage}[b]{0.86\columnwidth}\raggedright\strut
Finanzierung Grundversorgung
\strut\end{minipage}\tabularnewline
\midrule
\endhead
\begin{minipage}[t]{0.08\columnwidth}\raggedright\strut
AT
\strut\end{minipage} &
\begin{minipage}[t]{0.86\columnwidth}\raggedright\strut
Allowances (80\%) + FFS (contracted), FFS (non-contracted)
\strut\end{minipage}\tabularnewline
\begin{minipage}[t]{0.08\columnwidth}\raggedright\strut
BE
\strut\end{minipage} &
\begin{minipage}[t]{0.86\columnwidth}\raggedright\strut
FFS
\strut\end{minipage}\tabularnewline
\begin{minipage}[t]{0.08\columnwidth}\raggedright\strut
BG
\strut\end{minipage} &
\begin{minipage}[t]{0.86\columnwidth}\raggedright\strut
Capitation + bonuses
\strut\end{minipage}\tabularnewline
\begin{minipage}[t]{0.08\columnwidth}\raggedright\strut
CY
\strut\end{minipage} &
\begin{minipage}[t]{0.86\columnwidth}\raggedright\strut
Salary (public), FFS (private)
\strut\end{minipage}\tabularnewline
\begin{minipage}[t]{0.08\columnwidth}\raggedright\strut
CZ
\strut\end{minipage} &
\begin{minipage}[t]{0.86\columnwidth}\raggedright\strut
Age-weighted capitation + FFS
\strut\end{minipage}\tabularnewline
\begin{minipage}[t]{0.08\columnwidth}\raggedright\strut
DK
\strut\end{minipage} &
\begin{minipage}[t]{0.86\columnwidth}\raggedright\strut
Capitation + FFS
\strut\end{minipage}\tabularnewline
\begin{minipage}[t]{0.08\columnwidth}\raggedright\strut
EE
\strut\end{minipage} &
\begin{minipage}[t]{0.86\columnwidth}\raggedright\strut
Age-weighted capitation + FFS
\strut\end{minipage}\tabularnewline
\begin{minipage}[t]{0.08\columnwidth}\raggedright\strut
FI
\strut\end{minipage} &
\begin{minipage}[t]{0.86\columnwidth}\raggedright\strut
Salary + FFS or a mix of salary, capitation + FFS for personal doctors
(public), FFS (private)
\strut\end{minipage}\tabularnewline
\begin{minipage}[t]{0.08\columnwidth}\raggedright\strut
FR
\strut\end{minipage} &
\begin{minipage}[t]{0.86\columnwidth}\raggedright\strut
FFS
\strut\end{minipage}\tabularnewline
\begin{minipage}[t]{0.08\columnwidth}\raggedright\strut
DE
\strut\end{minipage} &
\begin{minipage}[t]{0.86\columnwidth}\raggedright\strut
FFS points
\strut\end{minipage}\tabularnewline
\begin{minipage}[t]{0.08\columnwidth}\raggedright\strut
EL
\strut\end{minipage} &
\begin{minipage}[t]{0.86\columnwidth}\raggedright\strut
Salary + FFS (public), FFS (private)
\strut\end{minipage}\tabularnewline
\begin{minipage}[t]{0.08\columnwidth}\raggedright\strut
HU
\strut\end{minipage} &
\begin{minipage}[t]{0.86\columnwidth}\raggedright\strut
Weighted capitation + adjustments based on provider characteristics
\strut\end{minipage}\tabularnewline
\begin{minipage}[t]{0.08\columnwidth}\raggedright\strut
IE
\strut\end{minipage} &
\begin{minipage}[t]{0.86\columnwidth}\raggedright\strut
Weighted capitation + FFS
\strut\end{minipage}\tabularnewline
\begin{minipage}[t]{0.08\columnwidth}\raggedright\strut
IT
\strut\end{minipage} &
\begin{minipage}[t]{0.86\columnwidth}\raggedright\strut
Capitation + FFS + PRP (also for paediatricians)
\strut\end{minipage}\tabularnewline
\begin{minipage}[t]{0.08\columnwidth}\raggedright\strut
LV
\strut\end{minipage} &
\begin{minipage}[t]{0.86\columnwidth}\raggedright\strut
Age-weighted capitation + FFS
\strut\end{minipage}\tabularnewline
\begin{minipage}[t]{0.08\columnwidth}\raggedright\strut
LT
\strut\end{minipage} &
\begin{minipage}[t]{0.86\columnwidth}\raggedright\strut
Age-weighted capitation
\strut\end{minipage}\tabularnewline
\begin{minipage}[t]{0.08\columnwidth}\raggedright\strut
LU
\strut\end{minipage} &
\begin{minipage}[t]{0.86\columnwidth}\raggedright\strut
FFS
\strut\end{minipage}\tabularnewline
\begin{minipage}[t]{0.08\columnwidth}\raggedright\strut
NL
\strut\end{minipage} &
\begin{minipage}[t]{0.86\columnwidth}\raggedright\strut
Salary
\strut\end{minipage}\tabularnewline
\begin{minipage}[t]{0.08\columnwidth}\raggedright\strut
MT
\strut\end{minipage} &
\begin{minipage}[t]{0.86\columnwidth}\raggedright\strut
Capitation + FFS
\strut\end{minipage}\tabularnewline
\begin{minipage}[t]{0.08\columnwidth}\raggedright\strut
PL
\strut\end{minipage} &
\begin{minipage}[t]{0.86\columnwidth}\raggedright\strut
Age-weighted capitation
\strut\end{minipage}\tabularnewline
\begin{minipage}[t]{0.08\columnwidth}\raggedright\strut
PT
\strut\end{minipage} &
\begin{minipage}[t]{0.86\columnwidth}\raggedright\strut
Salary (NHS) + capitation + PRP
\strut\end{minipage}\tabularnewline
\begin{minipage}[t]{0.08\columnwidth}\raggedright\strut
RO
\strut\end{minipage} &
\begin{minipage}[t]{0.86\columnwidth}\raggedright\strut
Age-weighted capitation + FFS (15\%)
\strut\end{minipage}\tabularnewline
\begin{minipage}[t]{0.08\columnwidth}\raggedright\strut
SI
\strut\end{minipage} &
\begin{minipage}[t]{0.86\columnwidth}\raggedright\strut
Age-weighted capitation + FFS
\strut\end{minipage}\tabularnewline
\begin{minipage}[t]{0.08\columnwidth}\raggedright\strut
SK
\strut\end{minipage} &
\begin{minipage}[t]{0.86\columnwidth}\raggedright\strut
Capitation + FFS (50\%)
\strut\end{minipage}\tabularnewline
\begin{minipage}[t]{0.08\columnwidth}\raggedright\strut
ES
\strut\end{minipage} &
\begin{minipage}[t]{0.86\columnwidth}\raggedright\strut
Salary + age-weighted capitation (15\%)
\strut\end{minipage}\tabularnewline
\begin{minipage}[t]{0.08\columnwidth}\raggedright\strut
SE
\strut\end{minipage} &
\begin{minipage}[t]{0.86\columnwidth}\raggedright\strut
Salary or capitation + some FFS
\strut\end{minipage}\tabularnewline
\begin{minipage}[t]{0.08\columnwidth}\raggedright\strut
UK-ENG
\strut\end{minipage} &
\begin{minipage}[t]{0.86\columnwidth}\raggedright\strut
Weighted capitation + FFS + PRP
\strut\end{minipage}\tabularnewline
\bottomrule
\end{longtable}

Quellen:,\footnote{Quellen: European Observatory on Health Systems and
  Policies Health Systems in Transition series reports.}
Abkürzungen:\footnote{Abkürzungen: FFS: Fee-for-service (payments); PRP:
  Performance-related pay.}

\subsection{Überlegungen zu einem optimalen Vergütungssystem aus den
Niederlanden}\label{uberlegungen-zu-einem-optimalen-vergutungssystem-aus-den-niederlanden}

Die Grundversorger üben in den Niederlanden eine Gatekeeping-Funktion
aus. Der erste Kontakt mit dem Gesundheitssystem ist zwingend bei einem
Hausarzt, welcher die weitere Versorgung koordiniert. Das
Vergütungssystem für Grundversoger wurde in den Niederlanden per
1.1.2006 reformiert. Die davor vorhandene Trennung zwischen öffentlich
und privat versicherten Personen, mit unterschiedlichen Zahlungssystemen
bei deren Behandlung, wurde aufgehoben. Neu kommt ebenfalls eine
Mischung aus Capitation und FFS zur Anwendung.\\
In Smolders und Seinen (2007) wird dieses neue System evaluiert, und
werden Vorschläge zur Optmierung bezüglich Minimierung der (gesamten)
Gesundheitsausgaben, sowie zu Qualität und Zugang zum Gesundheitssystem
präsentiert.\\
Das Wechsel zu einem Mischsystem aus Capitation und FFS wird als eine
Verbesserung zu den vorherigen reinen Capitation- (öffentlich
verischerte Personen), resp. FFS-Systeme (privat versicherte Personen)
angesehen. Vorher vorhandene unerwünschten Anreize konnten durch diesen
Wechsel vemindert werden.\\
Da ein Capitationsystem Anreize zu einer Unterproduktion mit sich
bringt, und ein FFS-System die Ärzte eher zu einer Überproduktion
motiviert, ethische Grundsätze der Ärzte aber eine Unterproduktion nur
begrenzt zulassen, soll das Vergütungssystem hauptsächlich aus einem
risikoadjustierten Capitationbeitrag, und zusätzlichen FFS-Beiträgen
bestehen - wie dies aktuell in den Niederlanden der Fall ist. Verbessert
werden könnte das System allenfalls dadurch, dass die dokumentierte
Verwendung von Protokollen und Guidelines über spezifische Positionen im
Einzelleistungstarif vergütet wird.\\
Zur Einführung von Budgets für Grundversorger für weitergehende
Behandlungen bei Spezialärzten oder stationären Einrichtungen, wird in
Smolders und Seinen (2007) aus den folgenden Gründen abgeraten:

\begin{itemize}
\tightlist
\item
  Das zusätzliche finanzielle Risiko für Grundversorger würde dazu
  führen, dass diese dafür eine Risikoprämie verlangen würden.
  Krankenversicherungen können dieses Risiko aufgrund des grösseren
  Versichertenkollektivs besser tragen.
\item
  Durch die sich konkurrenzierenden Krankenversicherungen gibt es in den
  Niederlanden bereits Leistungseinkäufer, welche versuchen hohe
  Qualität zu günstigen Preisen einzukaufen. Weiter Konkurrenz unter den
  Ärzten würde hier keinen Nutzen bringen.
\item
  Die Grundversorger haben eine kleinere Markmacht als
  Krankenversicherungen, auch wenn sie sich zusammenschliessen.
\item
  Budgets für weitergehende Behandlungen belasten potentiell das
  Verhältnis zu den Klienten.
\end{itemize}

Budgets für Medikamente werden ebenfalls nicht empfohlen. Auch hier
könnte Misstrauen der Klienten das Verhältnis zum Arzt beeinträchtigen.
Anstelle von Anreizen für tiefe Medikamentenverschreibungen werden
konkrete Einschränkungen vorgeschlagen. Beispielsweise könnte nur die
jeweils günstigste Variante eines bestimmten Medikamentes vergütet
werden. Auch die Verwendung von Generika kann vorausgesetzt werden, wo
dies möglich ist, um Kosten zu sparen.\\
Insgesamt wird die Mischung aus Capitation und FFS als ideal angesehen,
und es stellt sich lediglich die Frage, zu welchen Teilen diese beiden
Systeme zum Einkommen der Grundversorger beitragen sollen. Von stärkeren
finanziellen Anreizen für Grundversorger, z.B. durch Budgets, wird
abgeraten.

\subsection{Vergleich der Systeme in der Schweiz und den
Niederlanden}\label{vergleich-der-systeme-in-der-schweiz-und-den-niederlanden}

Die Vergütungssysteme der Grundversorger in der Schweiz und den
Niederlanden gleichen sich insofern, als dass beide eine Mischung aus
Capitation uns FFS beinhalten. Der grösste Unterschied besteht darin,
dass in den Niederlanden der Capitationbeitrag den grössten Anteil am
Einkommen der Ärzte hat. In der Schweiz beziehen Ärzte ihr Einkommen
haupsächlich aus FFS-Beiträgen; allfällige Capitationzahlungen
(üblicherweise im Zusammenhang mit einem Budget) spielen nur eine
marginale Rolle. Ärzte können sich in der Schweiz auch völlig vom
Capitationsystem fernhalten, und ihr Einkommen ausschliesslich aus
FFS-Beiträgen beziehen.\\
Da Capitationsysteme Anreize zu einer Unter- und FFS-Systeme zu einer
Überproduktion an Gesundheitsleistungen setzen, ist anzunehmen, dass das
schweizerische System mehr zu Überproduktion neigt, als dasjenige in den
Niederlanden.\\
Ein weiterer Unterschied in den beiden Ländern ist, dass in den
Niederlanden die Grundversorger gesetzlich vorgeschrieben eine
Gatekeeping-Rolle ausüben (kein direkter Zugang zu Spezialärzten). In
der Schweiz soll diese Rolle über die Budgetmitverantwortung in
Capitationverträgen gestärkt werden. Da solche Verträge wenig verbreitet
sind, und die Behandlung bei Spezialärzten als teuer gilt, ist dies ein
weiterer möglicher Grund, weshalb die Gesundheitskosten in der Schweiz
höher sind als in den Niederlanden.

\section{Schlussfolgerungen}\label{schlussfolgerungen}

Mit dem Mischssystem aus FFS und Capitation macht die Schweiz in der
Vergütung der hausärztlichen Leistungen das, was auch im benachbarten
Ausland üblich ist. Allerdings ist in der Schweiz nicht nur der
Reformbedarf gross, sondern auch das Potential für Verbesserungen im
heutigen System gegeben. Die Niederlande gibt hierbei durchaus ein
valables Vorbild ab, deren System zurzeit zusammen mit demjenigen in
Deutschland auch tatsächlich vom Departement des Inneren EDI analysiert
wird (EDI 2016).\\
Wie bereits im Masterplan ``Gesundheit 2020'' (EDI 2013) als mögliche
Massnahme zum Ziel 2.2 angetönt wird, könnte insbesondere die weitere
Verbreitung und höhere Gewichtung der Capitationbeiträge in der Schweiz
die bestehenden Probleme potentiell mildern. Namentlich könnte dies
einen dämpfenden Effekt auf die Menge der erbrachten ambulanten
Gesundheitsleistungen, und damit auf die totalen Gesundheitskosten,
haben. Zudem könnten dadurch die finanziellen Anreize für junge Ärzte,
sich zu Grundversorgern auszubilden und sich in ländlichen Regionen
anzusiedeln, vergrössert werden.\\
Ein weiterer Vorteil von Capitationsystemen ist, dass diese weniger an
Innovationen angepasst werden müssen, als dies bei FFS-Systemen der Fall
ist (in welchen jede abrechenbare Behandlungsmethode einzeln aufgeführt
werden muss). Durch eine stärkere Gewichtung der Capitation könnte so
auch der Druck in den Tarifverhandlungen verringert werden, wenn der
verhandlungsintensive Einzelleistungstarif einen kleineren Anteil am
Einkommen der Ärzte hat.\\
Politisch kaum eine Chance dürfte in der Schweiz die Einschränkung des
Zugangs zu Spezialärzten wie in den Niederlanden haben. Auch wenn
dadurch die Steigerung der Gesundheitskosten durch die Stärkung der
Hausarztmedizin und deren Gatekeeping-Rolle erwartet werden könnte,
haben vergangene Abstimmungen in der Schweiz gezeigt, dass ein solches
Anliegen beim Stimmvolk auf wenig Akzeptanz stossen würde.

\section*{Bibliography}\label{bibliography}
\addcontentsline{toc}{section}{Bibliography}

\hypertarget{refs}{}
\hypertarget{ref-ediux5fgesundheitux5f2013}{}
EDI. 2013. «Gesundheit 2020 - Die gesundheitspolitischen Prioritäten des
Bundesrates». Medienbericht. Eidg. Departement des Inneren.
\url{http://bit.ly/1orgI7g}.

\hypertarget{ref-ediux5fmengenwachstumux5f2016}{}
EDI. 2016. «Mengenwachstum im Gesundheitswesen eindämmen -- zusätzliche
Massnahmen nötig». \emph{Medienmitteilung}, Februar. Bern.
\url{http://bit.ly/2c5g0hg}.

\hypertarget{ref-forsterux5fsteigendeux5f2016}{}
Forster, Christof. 2016. «Steigende Gesundheitskosten: Berset will das
Wachstum drosseln». \emph{Neue Zürcher Zeitung}, Februar.
\url{http://bit.ly/2bUFaAY}.

\hypertarget{ref-nordwestschweizux5fschweizux5f2016}{}
Nordwestschweiz. 2016. «Der Schweiz fehlen über 2000 Hausärzte -- und es
werden mehr». \emph{Aargauer Zeitung}, März.
\url{http://bit.ly/2c9crUA}.

\hypertarget{ref-sangraux5ftarmedux5f2010}{}
Sangra, Emmanuel, und Claude Courbat. 2010. «TARMED - der Tarif für
ambulant erbrachte ärztliche Leistungen». Eidgenössische Finanzkontrolle
(EFK). \url{http://bit.ly/2c9dofG}.

\hypertarget{ref-smoldersux5foptimalux5f2007}{}
Smolders, Nicole, und Ingrid Seinen. 2007. «An optimal renumeration for
General Practitioners». Dutch Healthcare Authority (NZa).
\url{http://bit.ly/2ctP0Gr}.

\hypertarget{ref-thomsonux5ffinancingux5f2009}{}
Thomson, Sarah, Thomas Foubister, und Elias Mossialos. 2009.
\emph{Financing health care in the European Union: Challenges and policy
responses}. Observatory studies series, no. 17. Copenhagen: World Health
Organization on behalf of the European Observatory on Health Systems;
Policies. \url{http://bit.ly/1o5es2h}.

\end{document}
